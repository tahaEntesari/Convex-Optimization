\documentclass[a4paper]{article}
\usepackage{graphicx}
\usepackage{amsmath}
\usepackage{float}
\usepackage{geometry}
\usepackage{listings}
\usepackage{multicol}
\usepackage{float}
\usepackage{caption}
\usepackage{enumitem}
\usepackage[americanvoltages,fulldiodes,siunitx]{circuitikz}
\usepackage{wrapfig}
\usepackage{mathrsfs} % https://www.ctan.org/pkg/mathrsfs
\newcounter{MyCounter}
\usepackage{enumitem}
\usepackage{pgfplots}
\usepackage{subfig}	% ploting figures beside each other
\pgfplotsset{compat=1.12}

 \geometry{
 a4paper,
 total={170mm,257mm},
 left=20mm,
 top=20mm,
 }
\usepackage{fancyhdr}
\pagestyle{fancy}
\cfoot{(\space \space \space \space \textbf{\thepage}  \space \space \space)}
\renewcommand{\headrulewidth}{1pt}
\renewcommand{\footrulewidth}{1pt}
\usepackage{listings}
\usepackage{color} %red, green, blue, yellow, cyan, magenta, black, white
\definecolor{mygreen}{RGB}{28,172,0} % color values Red, Green, Blue
\definecolor{mylilas}{RGB}{170,55,241}


\begin{document}
	\lstset{language=Matlab,%
		%basicstyle=\color{red},
		breaklines=true,%
		morekeywords={matlab2tikz},
		keywordstyle=\color{blue},%
		morekeywords=[2]{1}, keywordstyle=[2]{\color{black}},
		identifierstyle=\color{black},%
		stringstyle=\color{mylilas},
		commentstyle=\color{mygreen},%
		showstringspaces=false,%without this there will be a symbol in the places where there is a space
		numbers=left,%
		numberstyle={\tiny \color{black}},% size of the numbers
		numbersep=9pt, % this defines how far the numbers are from the text
		emph=[1]{for,end,break},emphstyle=[1]\color{blue}, %some words to emphasise
		%emph=[2]{word1,word2}, emphstyle=[2]{style},    
	}
	
	





	\begin{center}
	\textbf{
	\\In the name of God
		}\\
	
	\vspace{2cm}
	\includegraphics[scale=.35]{logo1.png}\\
	\vspace{0.5cm}
	\begin{Large}
	\textbf{
	\\Sharif University of Technology
	\vspace{0.5cm}
	\\School of Electrical Engineering
	}
	\end{Large}
	\vspace{2cm}
	\begin{huge}
	\textbf{
	\\Convex Optimization
	\vspace{0.75cm}
‍
	\\Homework Nr. 3
	}
	\end{huge}
	\vspace{2cm}
	\begin{Large}
	\textbf{
	\\\textit{Dr}. Babazadeh
	\vspace{2cm}
	\\Taha Entesari
	\vspace{0.75cm}
	\\95101117
	}
	\end{Large}
	
\end{center}

\thispagestyle{empty}
\newpage
\begin{Large}
	The Simulation results are provided here. Oddly enough, non of the questions would reach a solution with certainty and are extremely dependent on the choice of random matrices that are generated at first, thus, the code is written in such a way to continue generating random matrices until a case is reached in which an answer is achieved for that problem. After reaching a solution, the said matrices are saved for future uses. 
	\section*{Problem 5}
	The CVX code for part a is simple but due to the random generation of matrices that is mentioned above, the result of the executed code would mostly be that the \textit{status} is \textit{infeasible} and the \textit{optimal value} is $ +\infty $. An instance of good matrices is saved and appended and available in the zip file.\\
	The code for part b is implemented in 2 ways; One is the solution provided by the textbook and the other is the less simplified one that I myself have derived. For the matrix set that is generated in part \textit{a}, both these methods, as expected, result in the same optimal value, and the mentioned optimal value is less than the one in part \textit{a} which suggests that there is a gap between the primal and the original problem. With the mentioned matrix, the result of part \textit{a} would be $ -0.46 $ whereas the optimal value achieved by part \textit{b} is $ -2.36 $.\\
	The question has asked to generate the matrices for different \textit{n} and compare the consumed \textit{cpu time} but the process of finding matrices that would result in solvable problems is very  time consuming and this process could not be completed.
	\section*{Problem 6}
	The dual problem of the relative entropy problem is implemented in this section. The problem in this problem is that almost always the \textit{status} would be \textit{unbounded} and the \textit{optimal value} would be $ +\infty $. Following the same procedure as the previous section, a set of good matrices are saved and are appended and available in the zip file. For the mentioned set of matrices the optimal value is $ -0.97442 $.


\end{Large}
\end{document}

